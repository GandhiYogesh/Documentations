\section*{Scalable Geometry Projection Topology Optimization method for anisotropic materials applied to additive manufacturing technology}

\paragraph{Introduction}The cost-effective commercially available Additive Manufacturing (AM) or 3D Printing (3DP) technologies eliminate many limitations that previously plagued the manufacturing of highly tailored structural performance for multi-functional and multi-physics applications. For example, continuous Fiber Filament Fabrication (CF4) technology provides a unique opportunity to control the fiber-reinforced composites' (FRC) anisotropic properties and effectively distribute the loads throughout the laminate to maximize the structure's strength and stiffness. This concept of performance-driven manufacturing is known as Design for Additive Manufacturing (DfAM). 

The optimization concept applied to FRC materials allows finding the optimized material distribution, the optimized orientation of fibre paths, and optimized geometric contours of the laminate. Topology Optimization (TO), one of the DfAM methods, is an iterative design tool to optimize a quantifiable objective while being intended to sustains loads, constraints, and boundary conditions. TO approaches can be summarized as follows: the homogenization method, the Solid Isotropic Material with Penalization (SIMP) method, the level set method, the Evolutionary Structural Optimization (ESO) method, Topology Derivatives and Phase Field. In addition, some are emerging TO methods for smooth boundary representation include the 'Metamorphic Development Method' (MDM) and the 'Moving Morphable Method' (MMM). The general architecture of TO starts with the definition of maximizing or minimizing a single or multi-target-objective function to fulfil a set of constraints such as volume, displacement, or frequency. Then, as part of an iterative process, design variables, Finite Element Analysis (FEA), sensitivity analysis, regularization, and optimization steps are repeated in this order until convergence is achieved.

\paragraph{Aim}From a geometry representation point of view, most of the above-mentioned TO approaches are developed within the pixel or node-based solution framework. Although this approach has made remarkable achievements, there are still some challenging issues that need further exploration. Firstly, the realization of the attained topology using pixel or node-based is not consistent because it is difficult to control the structural feature sizes, which is usually very important from manufacturing considerations. Secondly, the computational efforts in pixel or node-based topology optimization approaches are relatively large, especially considering anisotropic materials. 
The primary aim is to establish a direct link between structural topology optimization and manufacturability. The research utilizes optimal-layout techniques to produce structural layouts of geometric primitives prevalent in fabricating structures with continuous fiber reinforcements. The proposal is to employ the geometry projection method to produce designs made of fiber-reinforced geometric primitives. In this method, transversely isotropic bars with semicircular ends describes each of the geometric components.  The design space for the optimization consists of the endpoint locations of the bar's medial axes and their out-of-plane thicknesses. Then, the bar design is projected onto the analysis grid using a differentiable mapping from the design space, e.g., the endpoint positions, to a continuously varying density field, which indicates the fraction of solid material anywhere in the design space density-based topology optimization methods.  Primitives are then combined using a smooth maximum of the weighted projected densities. The size variables greatly facilitate the removal of components. They are penalized as the element densities in density-based TO and allow the optimizer to remove or reintroduce a primitive altogether. Furthermore, the proposed method naturally accommodates the imposition of several fixed-length scales, i.e., different bar widths.  

\paragraph{Methodology}
In contrast to these two-phase approaches (i.e., void and anisotropic bars), the research activity considered a three-phase approach wherein fiber fraction along with material fraction in a given design space. Such a composite structure is functionally graded because fiber is present in a fraction with voids and variable fiber density. The current focus is on sequential implementation, designing an isotropic-material matrix with voids, then inserting primitives selectively, followed by optimally orienting the primitives. The next stage combines all three steps by simultaneously interchanging three phases: isotropic matrix material, anisotropic bars, and voids. However, topology optimization fails for large-scale structures because it performs poorly due to a dramatic increase in the design variables that the optimization technique solves, especially considering anisotropic materials. Hence, the final goal is to implement a parallel geometry-based topology optimization method in three dimensions to make high-resolution and large-scale problems more attainable.

A 3D Topology optimization method needs reasonable time, especially when considering anisotropic materials. Thus, the algorithm needs to be scalable to thousands of processors and have high parallelism. The algorithm has been implemented with the parallel computing frameworks PETSc and MPI and should work with multiple processors. However, this does not mean that the algorithm communicates efficiently between processes and exploits all the opportunities for parallelization properly. Therefore, a scalability analysis needs to be carried out to run performance tests and gather information about how well the algorithm performs in a parallel environment. This endeavor aims to determine whether or not the parallel implementation of the algorithm is scalable to hundreds and even thousands of processors and is therefore scalable. Hence, the final goal is to implement a scalable analysis of the geometry-based topology optimization method in three dimensions to make high-resolution and large-scale problems more attainable.

\subsection{Result and Discussion}
 
\paragraph{Planning of research activity for next year}
The diagram summarized performed activities in the first academic year and outlined the plans for the next academic year. A thorough investigation of the literature on topology optimization, topology optimization of anisotropic material, and topology optimization for additive manufacturing helped me establish the gaps in the existing literature. Furthermore, it gave me detailed analytical and numerical aspects of the problem statement issued at the beginning of the DAST Ph.D. course. The objectives for the next term are to establish a numerical approach for Geometry Projection Topology Optimization (GPTO) and additively manufacture the design attained by the GTO. I will devote the starting months to devising and parallelizing the GPTO-based approach for an anisotropic material and simultaneously solving the optimization problem for three-phase material. Finally, we will convert this work into Structural Optimization and Multidisciplinary journal article in collaboration with Dr. Alejandro M. Aragòn (TUDelft) and Dr. Julian Norato Escobar (University of Connecticut). The last few months of the second year will focus on integrating additive manufacturing constraints into the developed GPTO algorithm and reiterating the features of the algorithm to minimize the user interactions and automate the DfAM process. Finally, the outcome will be published as an article in collaborative work with the University of Padova. With this brief overview on the research aspect, I believe in consistent documentation of research activities performed during the Ph.D. course. Thus, great emphasis is given to reporting important detail to ease out the process of tractability of my work to other researchers and students. Finally, the educational code on topology optimization is another paper I aim to publish to contribute to the research community offered me during my Ph.D. tenure.  
